\cqusetup{
%	************	注意	************
%	* 1. \cqusetup{}中不能出现全空的行,如果需要全空行请在行首注释
%	* 2. 不需要的配置信息可以放心地坐视不理、留空、删除或注释(都不会有影响)
%	*
%	********************************
% ===================
%	论文的中英文题目
% ===================
  ctitle = {车载信息物理融合系统关键技术研究},
  etitle = {Research on Key Techniques for Vehicular Cyber-Physical Systems},
% ===================
% 作者部分的信息
% \secretize{}为盲审标记点,在打开盲审开关时内容会自动被替换为***输出,盲审开关默认关闭
% ===================
  cauthor = \secretize{许新操},	% 你的姓名,以下每项都以英文逗号结束
  eauthor = \secretize{Xincao~Xu},	% 姓名拼音,~代表不会断行的空格
  studentid = \secretize{},	% 仅本科生,学号
  csupervisor = \secretize{刘~~~~凯~~~~教授},	% 导师的姓名
  esupervisor = \secretize{{Prof.~Kai Liu}},	% 导师的姓名拼音
  cassistsupervisor = \secretize{}, % 本科生可选,助理指导教师姓名,不用时请留空为{}
  cextrasupervisor = \secretize{}, % 本科生可选,校外指导教师姓名,不用时请留空为{}
  eassistsupervisor = \secretize{}, % 本科生可选,助理指导教师或/和校外指导教师姓名拼音,不用时请留空为{}
  cpsupervisor = \secretize{}, % 仅专硕,兼职导师姓名
  epsupervisor = \secretize{},	% 仅专硕,兼职导师姓名拼音
  cclass = \rmfamily{2023}年\rmfamily{6}月,	% 博士生和学硕填学科门类,学硕填学科类型
  research_direction = \zihao{3}{车联网},
  edgree = {},	% 专硕填Professional Degree,其他按实情填写
% 提示:如果内容太长,可以用\zihao{}命令控制字号,作用范围:{}内
  cmajor = 工~~~~学,	% 专硕不需填,填写专业名称
  emajor = , % % 专硕不需填,填写专业英文名称
  cmajora = \zihao{3}{计算机科学与技术},
  cmajorb = \zihao{3}{车联网},
  cmajorc = \secretize{******~~~~教授},
%  cmajord = 2023年6月,
% ===================
% 底部的学院名称和日期
% ===================
  cdepartment = ,	%学院名称
  edepartment = ,	%学院英文名称
% ===================
% 封面的日期可以自动生成(注释掉时),也可以解除注释手动指定,例如:二〇一六年五月
% ===================
%	mycdate = {2023年6月},
%	myedate = {June 2023},
}% End of \cqusetup
% ===================
%
% 论文的摘要
%
% ===================
\begin{cabstract}	% 中文摘要

随着感知模式、通讯技术与计算范式的发展,传统汽车正朝着智能化、网联化、协同化方向迅猛演进。
以智能网联汽车为抓手,车联网驱动的智能交通系统(Intelligent Transportation System, ITS)有望于实现更加安全、高效、可持续发展的交通运输。
车载信息物理融合系统(Vehicular Cyber-Physical System, VCPS)是实现ITS应用的基础和关键。
然而,车联网的高异构、高动态、分布式的特征和ITS应用的多元化需求都给实现 VCPS 带来了巨大的挑战。
首先,面向高动态异构车联网研究创新的服务架构并建立高效的数据感知与质量评估模型是 VCPS 的架构基础与驱动核心。
其次,面对动态异构节点资源,提出先进的任务调度与资源分配是进一步优化 VCPS 服务质量的技术支持。
再次,面向多元 ITS 应用需求,设计系统质量-开销均衡策略是实现高质量可扩展 VCPS 的理论保障。
最后,面向真实复杂车联网环境,基于 VCPS 设计并实现原型系统是验证 VCPS 的必要手段。
因此,本文面向车载信息物理融合系统,从质量指标设计、协同资源优化、质量-开销均衡,以及原型系统实现四个方面进行了理论创新,主要创新成果包括:

\circled{1} 基于分层车联网架构的车载信息物理融合质量指标设计与优化。
首先,提出了面向下一代车联网的分层服务架构,旨在车联网中有机融合软件定义网络和移动边缘计算范式,并最大化其在信息服务方面的协同效应。
在此基础上,提出了车辆协同感知和异质信息融合场景,其中边缘节点融合由车辆协同感知的异构信息并构建逻辑视图。
其次,基于多类M/G/1优先队列建立了感知信息排队模型,并基于异质信息的时效性、完整性和一致性对车载信息物理融合质量进行建模。
具体地,设计了一个新颖的评估指标 Age of View(AoV)来定量评估边缘视图的质量。 
再次,提出了一个基于差分奖励的多智能体深度强化学习(Multi-Agent Difference-Reward-based Deep Reinforcement Learning, MADR)算法来最大化 AoV。
特别地,系统状态包括车辆感知信息、边缘节点缓存信息和视图需求。
车辆动作空间包括信息感知频率和上传优先级。
边缘节点根据预测的车辆轨迹和视图需求给车辆分配车与基础设施通信(Vehicle-to-Infrastructure, V2I)带宽。
最后,构建了仿真实验模型并进行了全面的性能评估,证明了MADR算法的优越性。

\circled{2} 面向车载信息物理融合的通信与计算资源协同优化关键技术。
首先,提出了基于非正交多址接入(Non-Orthogonal Multiple Access, NOMA)技术的 VEC 架构,其通过异构边缘节点协同进行实时任务处理。
其次,考虑 NOMA 车联网中域内和域间的干扰并建立了V2I传输模型,并形式化定义了协同资源优化问题,其旨在最大化服务率。
再次,将协同资源优化分解为两个子问题,即任务卸载和资源分配。
特别地,将任务卸载子问题建模为严格势博弈,并提出了多智能体分布式深度确定性策略梯度(Multi-Agent Distributed Distributional Deep Deterministic Policy Gradient, MAD4PG)算法来实现纳什均衡;
资源分配子问题被分解为两个独立的凸优化问题,并使用基于梯度的迭代方法和KKT条件得到了最优解。
最后,构建了仿真模型并进行了全面的性能评估,证明了MAD4PG算法的优越性。

\circled{3} 面向车载信息物理融合的质量-开销均衡优化关键技术。
首先,建立了协同感知模型和 V2I 上传模型,考虑边缘视图中异质信息的及时性、一致性建立了车载信息物理融合质量模型,并考虑视图信息冗余度、信息感知能耗,以及传输能耗建立了车载信息物理融合开销模型。
在此基础上,形式化定义了双目标优化问题,以最大化VCPS质量和最小化开销。
其次,提出了多智能体多目标深度强化学习(Multi-Agent Multi-Objective Deep Reinforcement Learning, MAMO)算法来实现质量-开销均衡,其中提出了决斗评论家网络,其基于状态价值和动作优势来评估智能体动作。
最后,进行了全面的性能评估,证明了MAMO算法的优越性。

\circled{4} 基于车载信息物理融合的超视距碰撞预警原型系统设计与实现。
首先,介绍了超视距(None-Light-of-Sight, NLOS)碰撞预警场景,其中交叉路口中的车辆由于视线遮挡而具有潜在碰撞风险,而传统基于视距的碰撞预警已不适用。
其次,提出了 V2I 应用层时延拟合模型和数据包丢失监测机制,并提出基于视图修正的碰撞预警算法,通过丢包检测与时延补偿实现更加精准的逻辑视图以提高碰撞预警性能。
再次,建立了基于真实车辆轨迹的仿真实验模型并进行了全面性能评估,证明了所提碰撞预警算法的优越性。
最后,搭建了基于车载终端和路侧设备的硬件在环试验平台,并搭建了基于无人小车的验证平台,进一步在真实的车联网环境中验证了超视距碰撞预警原型系统的有效性。

\end{cabstract}
% 中文关键词,请使用英文逗号分隔:
\ckeywords{车载信息物理融合系统,异构车联网,车载边缘计算,资源优化,多智能体深度强化学习}

\begin{eabstract}	% 英文摘要

With the development of sensing patterns, communication technologies, and computing paradigms, traditional vehicles are rapidly evolving towards intelligence, networking, and collaboration. 
By leveraging intelligent connected cars as a starting point, the intelligent transportation system (ITS) driven by vehicular networks is expected to realize safer, more efficient, and sustainable transportation.
The vehicular cyber-physical system (VCPS) in vehicular networks is the foundation and key to realizing the ITS applications. 
However, the inherent features of vehicular networks and the diverse demands of ITS applications have brought tremendous challenges to the realization of VCPS. 
First, to face the high-dynamic heterogeneous vehicular networks, it is necessary to integrate different computing paradigms and service architectures as the architecture basis of VCPS. 
Second, facing the distributed and time-varying physical environment, effective data sensing and modeling evaluation drive the data support of VCPS. 
Third, facing the dynamic and heterogeneous node resources, efficient task scheduling and resource allocation drive the technical support of VCPS. 
Fourth, facing the diverse ITS application requirements, achieving the tradeoff of system quality-cost and meeting the demands of differential applications are the theoretical guarantees of driving VCPS. 
Finally, facing the real and complex vehicular networks, designing and implementing a prototype system based on VCPS is the system verification of driving VCPS. 
Therefore, this thesis conducts research on the vehicular cyber-physical systems for heterogeneous vehicular networks from five aspects: service architecture integration, evaluation metric design, collaborative resource optimization, quality-cost tradeoff, and prototype system implementation. 
The main research contributions are as follows:

\circled{1} Research on the architecture of heterogeneous vehicular networks based on software-defined network and edge computing.
First, a hierarchical service architecture for the next generation vehicular networks is proposed in order to organically integrate software-defined network and edge computing paradigms  and maximize their synergistic effects in information services. 
Specifically, a hierarchical service architecture is designed, consisting of application layer, control layer, virtualization layer, and data layer. 
It achieves logically centralized control by separating the control plane and the data plane, promotes adaptive resource allocation based on network function virtualization and network slicing, and enhances the responsiveness, reliability, and scalability of the system by using the network, computing, and storage capabilities of the distributed service of edge computing. 
Second, the emerging challenges faced by this architecture are further analyzed, and future research directions are discussed by proposing a cross-layer protocol stack. 
Finally, in order to demonstrate the feasibility of the architecture, two case studies are implemented in a real vehicular network environment, which not only proves the tremendous potential of the proposed architecture, but also provides inspiration for the development of the next generation of ITS.

\circled{2} Research on the evaluation metric (Age of view) design and optimization for VCPS in vehicular edge computing.
First, a cooperative sensing and heterogeneous information fusion architecture for vehicular edge computing (VEC) is proposed, in which edge nodes fuse the heterogeneous information sensed by vehicle collaboratively and construct a logical view. 
Secondly, a queuing model for sensing information is established based on the multi-class M/G/1 priority queue, and the quality of vehicular cyber-physical system is modeled based on the timeliness, completeness, and consistency of heterogeneous information. 
Specifically, a novel evaluation metric, Age of View (AoV), is designed to quantitatively evaluate the quality of the logical view. 
Third, a multi-agent difference-reward-based deep reinforcement learning (MADR) algorithm is proposed to maximize AoV. 
In particular, the system state includes vehicle sensing information, edge node cached information, and view requirements. 
The vehicle action space includes information sensing frequency and upload priority. 
Based on the predicted vehicle trajectories and view requirements, the edge node allocates vehicle-to-infrastructure (V2I) bandwidth to  vehicles. 
Finally, a simulation model is constructed for comprehensive performance evaluation, which proves the superiority of the MADR algorithm.

\circled{3} Research on the cooperative optimization for heterogeneous resources in NOMA-based vehicular edge computing.
First, a VEC architecture based on Non-Orthogonal Multiple Access (NOMA) technology is proposed, which performs real-time task processing via the cooperation of heterogeneous edge nodes. 
Second, considering the interference within and between edge domains in NOMA-based vehicular networks, a V2I transmission model is established and the collaborative resource optimization problem is formally defined, which aims to maximize the service ratio. 
Third, the cooperative resource optimization problem is decomposed into two sub-problems: task offloading and resource allocation. 
In particular, the task offloading subproblem is modeled as an exact potential game, and a multi-agent distributed distributional deep deterministic policy gradient (MAD4PG) algorithm is proposed to achieve Nash equilibrium; The resource allocation subproblem is decomposed into two independent convex optimization problems, and the optimal solution is obtained using a gradient-based iterative method and the KKT conditions. 
Finally, a simulation model is constructed for comprehensive performance evaluation, which proves the superiority of the MAD4PG algorithm.


\circled{4} Research on quality-cost tradeoff optimization for vehicular cyber-physical system.
First, a vehicle cyber-physical fusion architecture is proposed, in which heterogeneous information is sensed by vehicles and uploaded to edge nodes via vehicle-to-infrastructure (V2I) communications. 
The sensing information is fused and modeled to obtain digital twins of various elements of the vehicular network, and a logical view is further constructed to reflect the real-time status of the physical vehicular network environment.
Second, cooperative sensing model and V2I uploading model are established, and considering the timeliness, consistency, and redundancy of digital twins, as well as the sensing and transmission costs, a VCPS quality model and cost model are established. 
On this basis, a bi-objective optimization problem is formally defined to maximize VCPS quality and minimize cost.
Third, a multi-agent multi-objective deep reinforcement learning (MAMO) algorithm is proposed to achieve quality-cost tradeoff, in which a dueling critic network is proposed to evaluate agent actions based on the state-value and action-advantage.
Finally, a comprehensive performance evaluation is conducted, demonstrating the superiority of the proposed MAMO algorithm.

\circled{5} Design and implementation of a non-line-of-sight collision warning prototype system based on VCPS.
First, the none-line-of-sight (NLOS) collision warning scenario is introduced, in which vehicles in intersections have potential collision risks due to LOS obstructions, and traditional LOS-based collision warnings are no longer applicable.
Second, a V2I application-layer delay fitting model and packet loss detection mechanism are proposed, along with a view calibration based collision warning algorithm. 
The algorithm achieves a more accurate logical view by utilizing packet loss detection and delay compensation, thus improving collision warning performance.
Third, a simulation experiment model based on real vehicle trajectories is established and comprehensive performance evaluations are conducted, proving the superiority of the proposed collision warning algorithm.
Lastly, using cellular vehicle-to-everything (C-V2X) onboard units and roadside units, a hardware-in-the-loop testing platform is built, and a validation platform based on unmanned vehicles is constructed. 
The effectiveness of the proposed algorithm and system is verified in a real vehicular network environment.
 
\end{eabstract}
% 英文关键词,请使用英文逗号分隔,关键词内可以空格:
\ekeywords{Vehicular cyber-physical systems, Heterogeneous vehicular networks, Metric design, Optimization method, Deep reinforcement learning
}

% 封面和摘要配置完成
