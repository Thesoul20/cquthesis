\chapter{总结与展望}\label{section 6}

车联网是通过网络技术将车辆、道路基础设施以及其他联网设备连接起来,实现车辆信息的互联互通及共享的新兴技术。
车联网有助于提高行车安全、促进交通顺畅、降低能源消耗、提升驾驶体验和推动智慧城市等方面,对于我国现代化城市建设和汽车产业发展都具有重要的推动作用。
同时,车载信息物理融合系统已成为支撑车联网中各类智能交通系统应用的关键。
本文致力于从服务架构融合、评估指标设计、资源协同优化、质量-开销均衡和原型系统实现五个方面协同驱动面向异构车联网的车载信息物理融合系统。
首先,面向高动态异构车联网,融合不同的计算范式与服务架构是实现车载信息物理融合的基础。
其次,面向分布式时变物理环境,有效的数据获取与建模评估是驱动车载信息物理融合的核心。
再次,面对动态异构节点资源,高效的任务调度与资源分配是进一步提升ITS服务质量的关键。
另外,面向多元智能交通系统应用需求,满足差异性的系统质量与系统开销需求是驱动车载信息物理融合的另一关键。
最后,面向复杂的真实车联网环境,基于车载信息物理融合系统进行有效设计并实现具体系统原型是具有挑战的。
本文的主要贡献如下:

\circled{1} 基于分层车联网架构的车载信息物理融合质量指标设计与优化。
首先,将SND与移动边缘计算融入于车联网,设计了包括应用层、控制层、虚拟层以及数据层的车联网分层服务架构,旨在最大化SDN逻辑上集中控制与分布式边缘计算服务的协同效应。
其次,提出了协同感知与异质信息融合场景,其中边缘节点通过融合车辆感知信息构建边缘视图以反映实时车联网环境。
其次,建立了基于多类M/G/1优先队列的感知信息排队模型,并进一步基于视图中异质信息的时效性、完整性和一致性建模,设计了崭新的车载信息物理融合质量指标。
在此基础上,形式化定义了视图质量最大化问题。
再次,提出了基于差分奖励的多智能体强化学习算法,其中车辆动作空间包括信息感知频率与上传优先级,边缘节点基于车辆预测轨迹和视图需求分配 V2I 带宽,并通过基于差分奖励的信用分配机制评估车辆对于视图构建的贡献。
最后,仿真实验结果表明,所提MADR算法能有效提高车载信息物理融合质量。

\circled{2} 面向车载信息物理融合的通信与计算资源协同优化关键技术。
首先,提出了协同通信与计算场景,其中边缘节点协同分配通信与计算资源以对车载信息物理融合任务进行实时处理。
其次,考虑了采用非正交多址接入技术的车联网中边缘内和边缘间干扰,建立了V2I传输模型,
在此基础上,形式化定义了最大化任务服务率的协同资源优化问题。
再次,提出了基于博弈的多智能体强化学习算法,具体地,将协同资源优化问题分解为任务卸载与资源分配子问题,其中将任务卸载子问题建模为严格势博弈模型,并进一步使用多智能体分布式深度确定性梯度策略算法来实现纳什均衡,另一方面,将资源分配子问题分解为两个独立凸优化问题,并分别使用基于梯度的迭代方法和KKT条件得到最优解。
最后,仿真实验结果表明,所提MAGT算法能能有效提高车载信息物理融合任务完成率。

\circled{3}  面向车载信息物理融合的质量-开销均衡优化关键技术。
首先,提出了车载信息物理融合质量-开销均衡场景,其中基于车辆的分布式感知和 V2I 通信协同上传,边缘节点构建边缘视图并考虑构建质量和构建能源消耗。
其次,考虑了视图的及时性、一致性和冗余度,以及感知与传输开销,建立了车载信息物理融合系统质量和开销模型,并进一步形式化定义了双目标优化问题,其目标为最大化VCPS质量且最小化VCPS开销。
再次,提出了面向多目标的多智能体强化学习算法,其中提出了决斗评论家网络,基于状态价值和动作优势来评估智能体的动作。
具体地,系统奖励为一维向量,其中包含了实现的VCPS质量和VCPS利润,并基于随机奖励权重与智能体随机动作来获得智能体动作对于随机动作的平均优势。
最后,仿真实验结果表明,MAMO算法能有效实现车载信息物理融合质量和车载信息物理融合开销的均衡。

\circled{4} 面向车载信息物理融合的超视距碰撞预警原型系统设计及实现。
首先,提出了超视距碰撞预警场景,其中交叉路口中车辆由于非视距问题存在潜在碰撞风险。
其次,提出基于视图修正的碰撞预警算法,具体地,基于现场测试获得的V2I应用层传输时延数据,建立了基于稳定分布的V2I时延拟合模型,并设计了基于数据上传频率和车辆状态历史信息的无线传输丢包检测机制。
通过时延估计和丢包检测对视图进行修正,以提供更加准确实时的碰撞预警服务。
再次,基于真实车辆轨迹搭建了实验仿真平台,仿真实验结果表明,所提 VCCW 算法能有效提高碰撞预警的查准率和查全率。
最后,搭建了基于C-V2X的硬件在环试验平台,对C-V2X端到端时延进行了分析,进一步基于无人小车搭建了验证平台,并在真实复杂车联网环境中实现了超视距碰撞预警原型系统,验证了VCCW算法的有效性。

本文主要针对车载信息物理融合系统关键技术开展了研究,并取得了一定的成果。
但作为探索VCPS的早期阶段,本文工作无法完全解决所有挑战,有待进一步探索和解决。
在后续工作中,将进一步研究边缘节点之间的合作,以扩大支持的ITS应用,并提高整体系统的性能。
其次,通过考虑车辆移动性和边缘节点之间协同计算的内在关系来进一步提高系统性能。
此外,将考虑车联网端边云架构,通过利用车辆、边缘节点和云协同来提高性能。
