\chapter{附\hskip\ccwd{}\hskip\ccwd{}录}
\section{作者在攻读学位期间的论文目录}

%下面是盲审标记\cs{secretize}的用法,记得去\textsf{main.tex}开启盲审开关看效果:

\circled{1}已发表论文

\begin{enumerate}
	\item 作者\textbf{\secretize{Xincao Xu}}, 导师\secretize{Kai Liu}, Penglin Dai, Feiyu Jin, Hualing Ren, Choujun Zhan and Songtao Guo. Joint Task Offloading and Resource Optimization in NOMA-Based Vehicular Edge Computing: A Game-Theoretic DRL Approach[J]. \textit{Journal of Systems Architecture}, 2023, 134: 102780. 影响因子: 5.836(2021), 4.497(5年) (JCR 1区,中科院SCI 2区) (对应本文第四章)
	\item 作者\textbf{\secretize{许新操}}, 导师\secretize{刘凯}, 刘春晖, 蒋豪, 郭松涛, 吴巍炜. 基于势博弈的车载边缘计算信道分配方法[J]. \textit{电子学报}, 2021,49(5): 851-860. (EI 索引,CCF T1类中文高质量科技期刊)
	\item 作者\textbf{\secretize{Xincao Xu}}, 导师\secretize{Kai Liu}, Ke Xiao, Liang Feng, Zhou Wu and Songtao Guo. Vehicular Fog Computing Enabled Real-time Collision Warning via Trajectory Calibration[J]. \textit{Mobile Networks and Applications}, 2019, 25(6): 2482-2494. 影响因子: 2.602(2019), 2.76(5年) (JCR 2区,中科院SCI 3区) (对应本文第六章)
	\item 导师\secretize{Kai Liu}, 作者\textbf{\secretize{Xincao Xu}}, Mengliang Chen, Bingyi Liu, Libing Wu and Victor Lee. A Hierarchical Architecture for the Future Internet of Vehicles[J]. \textit{IEEE Communications Magazine}, 2018, 57(7): 41-47. 影响因子: 10.356(2018), 12.091(5年) (JCR 1区,中科院SCI 1区) (对应本文第二章)
	\item 作者\textbf{\secretize{Xincao Xu}}, 导师\secretize{Kai Liu}, Qisen Zhang, Hao Jiang, Ke Xiao and Jiangtao Luo. Age of View: A New Metric for Evaluating Heterogeneous Information Fusion in Vehicular Cyber-Physical Systems[C]. \textit{Proceedings of IEEE International Conference on Intelligent Transportation Systems (IEEE ITSC’22)}, Macau, China, October 8-12, 2022. (EI 检索)
	\item 作者\textbf{\secretize{许新操}}, 周易, 导师\secretize{刘凯}, 向朝参, 李艳涛, 郭松涛. 车载雾计算环境中基于势博弈的分布式信道分配[C]. \textit{第十四届中国物联网学术会议(CWSN’20)}, 中国敦煌, 2020/9/18-9/21.
	\item 作者\textbf{\secretize{Xincao Xu}}, 导师\secretize{Kai Liu}, Ke Xiao, Hualing Ren, Liang Feng and Chao Chen. Design and Implementation of a Fog Computing Based Collision Warning System in VANETs[C]. \textit{Proceedings of IEEE International Symposium on Product Compliance Engineering-Asia (IEEE ISPCE-CN’18)}, Hong Kong/Shengzhen, December 5-7, 2018. (EI 检索)
	\item Chunhui Liu, 导师\secretize{Kai Liu}, Hualing Ren, 作者\textbf{\secretize{Xincao Xu}}, Ruitao Xie and Jingjing Cao. RtDS: Real-time Distributed Strategy for Multi-period Task Offloading in Vehicular Edge Computing Environment[J]. \textit{Neural Computing and Applications}, to appear. 影响因子: 5.606(2020), 5.573(5年) (JCR 1区,中科院SCI 2区)
	\item Ke Xiao, 导师\secretize{Kai Liu}, 作者\textbf{\secretize{Xincao Xu}}, Liang Feng, Zhou Wu and Qiangwei Zhao. Cooperative Coding and Caching Scheduling via Binary Particle Swarm Optimization in Software Defined Vehicular Networks[J]. \textit{Neural Computing and Applications}, 2021, 33(5): 1467-1478. 影响因子: 5.606(2020), 5.573(5年) (JCR 1区,中科院SCI 2区)
	\item Ke Xiao, 导师\secretize{Kai Liu}, 作者\textbf{\secretize{Xincao Xu}}, Yi Zhou and Liang Feng. Efficient Fog-assisted Heterogeneous Data Services in Software Defined VANETs[J]. \textit{Journal of Ambient Intelligence and Humanized Computing}, 2021, 12(1): 261-273. 影响因子: 7.104 (2020), 6.163 (5年) (JCR 2区,中科院SCI 3区)
	\item Chunhui Liu, 导师\secretize{Kai Liu}, 作者\textbf{\secretize{Xincao Xu}}, Hualing Ren, Feiyu Jin and Songtao Guo. Real-time Task Offloading for Data and Computation Intensive Services in Vehicular Fog Computing Environments[C]. \textit{Proceedings of IEEE International Conference on Mobility, Sensing and Networking (IEEE MSN’20)}, Tokyo, Japan, December 17-19, 2020. (EI 检索,CCF C类国际会议)
	\item Yi Zhou, 导师\secretize{Kai Liu}, 作者\textbf{\secretize{Xincao Xu}}, Chunhui Liu, Liang Feng and Chao Chen. Multi-period Distributed Delay-sensitive Tasks Offloading in a Two-layer Vehicular Fog Computing Architecture[C]. \textit{Proceedings of International Conference on Neural Computing and Applications (NCAA’20)}, Shenzhen, China, July 3-6, 2020. (EI 检索)
	\item Yi Zhou, 导师\secretize{Kai Liu}, 作者\textbf{\secretize{Xincao Xu}}, Songtao Guo, Zhou Wu, Victor Lee and Sang Son. Distributed Scheduling for Time-Critical Tasks in a Two-layer Vehicular Fog Computing Architecture[C]. \textit{Proceedings of IEEE Consumer Communications and Networking Conference (IEEE CCNC’20)}, Las Vegas, USA, January 11-14, 2020. (EI 检索)
\end{enumerate}

\circled{2}已投稿论文

\begin{enumerate}
	\item 作者\textbf{\secretize{Xincao Xu}}, 导师\secretize{Kai Liu}, Penglin Dai, Ruitao Xie, and Jiangtao Luo. Cooperative Sensing and Heterogeneous Information Fusion in VCPS: A Multi-agent Deep Reinforcement Learning Approach [J]. \textit{IEEE Transactions on Intelligent Transportation Systems}, Major Revisions. 影响因子: 9.551 (2021), 9.502 (5年) (JCR 1区, 中科院SCI 1区) (对应本文第三章)
	\item 导师\secretize{Kai Liu},作者\textbf{\secretize{Xincao Xu}}, Penglin Dai, and Biwen Chen. Cooperative Sensing and Uploading for Quality-Cost Tradeoff of Digital Twins in VEC [J]. \textit{IEEE Transactions on Consumer Electronics}, Major Revisions. 影响因子: 4.414 (2021), 3.565 (5年) (JCR 2区, 中科院SCI 2区) (对应本文第五章) 
\end{enumerate}

\section{作者在攻读学位期间取得的科研成果目录}
\begin{enumerate}
	\item 作者\textbf{\secretize{许新操}}, 导师\secretize{刘凯}, 李东. 一种针对软件定义车联网的控制平面视图构建方法. 发明专利. ZL202110591822.1.
	\item 导师\secretize{刘凯}, 张浪, 作者\textbf{\secretize{许新操}}, 任华玲, 周易. 一种基于边缘计算的盲区车辆碰撞预警方法. 发明专利. ZL201910418745.2.
	\item 任华玲, 导师\secretize{刘凯}, 陈梦良, 周易, 作者\textbf{\secretize{许新操}}. 一种基于雾计算的信息采集、计算、传输架构. 发明专利. ZL201910146357.3.
\end{enumerate}

\section{作者在攻读学位期间参与的科研项目目录}
\begin{enumerate}
	\item 国家自然科学基金面上项目,“面向车联网边缘智能的计算模型部署与协同跨域优化”,项目编号: 62172064,2022/01–2025/12(项目参与人员)
	\item 国家自然科学基金面上项目,“面向大规模数据服务的异构融合车联网架构与协议研究”,项目编号: 61872049,2019/01–2022/12(项目参与人员)
\end{enumerate}

\section{学位论文相关代码}
\begin{enumerate}
	\item 滴滴 GAIA 公开数据集处理源代码\\https://github.com/neardws/Vehicular-Trajectories-Processing-for-Didi-Open-Data
	\item 基于差分奖励的多智能体强化学习源代码\\https://github.com/neardws/Multi-Agent-Deep-Reinforcement-Learning
	\item 基于博弈理论的多智能体强化学习源代码\\https://github.com/neardws/Game-Theoretic-Deep-Reinforcement-Learning
	\item 基于多目标的多智能体强化学习源代码\\https://github.com/neardws/MAMO-Deep-Reinforcement-Learning
	\item 基于视图校准的碰撞预警源代码\\https://github.com/neardws/fog-computing-based-collision-warning-system
	\item 基于C-V2X通信的碰撞预警原型系统实现源代码\\https://github.com/neardws/V2X-based-Collision-Warning
\end{enumerate}

\section{多类M/G/1优先队列排队时延上界分析}\label{appendix e}

在车辆$v$中类型为$\operatorname{type}_d$的信息排队时间的方差由公式\ref{equ e.1}得到,其中$\alpha_{d, v}^t$、$\beta_{d, v}^t$和$\gamma_{d, v}^t$分别是信息$d$的传输时间的平均值和有限二和三阶矩。
\begin{align}
	{Var}(\operatorname{q}_{d, v}^t) &= \frac{\beta_{d, v}^t}{(1- \rho_{d, v}^{t})^2} + \frac{\alpha_{d, v}^t \sum\limits_{\forall d^* \in D_{d, v}^t} \lambda_{d^*, v}^t \beta_{d^*, v}^t}{(1- \rho_{d, v}^{t})^3} + \frac{\lambda_{d, v}^{t} \gamma_{d, v}^t + \sum\limits_{\forall d^* \in D_{d, v}^t} \lambda_{d^*, v}^t \gamma_{d^*, v}^t}{3(1- \rho_{d, v}^{t})^2(1-\rho_{d, v}^{t} - \lambda_{d, v}^{t}  \alpha_{d, v}^t)} \notag \\ 
	& + \frac{(\lambda_{d, v}^{t} \beta_{d, v}^t + \sum\limits_{\forall d^* \in D_{d, v}^t} \lambda_{d^*, v}^t \beta_{d^*, v}^t) \sum\limits_{\forall d^* \in D_{d, v}^t} \lambda_{d^*, v}^t \beta_{d^*, v}^t }{2(1- \rho_{d, v}^{t})^3(1-\rho_{d, v}^{t} - \lambda_{d, v}^{t}  \alpha_{d, v}^t)} - \beta_{d, v}^t \notag \\
	& + \frac{(\lambda_{d, v}^{t} \beta_{d, v}^t + \sum\limits_{\forall d^* \in D_{d, v}^t} \lambda_{d^*, v}^t \beta_{d^*, v}^t)^2}{4(1- \rho_{d, v}^{t})^2(1-\rho_{d, v}^{t} - \lambda_{d, v}^{t}  \alpha_{d, v}^t)^2}
\label{equ e.1}
\end{align}
根据切比雪夫不等式,我们有以下不等式
\begin{equation}
	\operatorname{Pr}(|\operatorname{q}_{d, v}^t - \operatorname{\bar{q}}_{d, v}^t| > j \sqrt{{Var}(\operatorname{q}_{d, v}^t)}) \leq \frac{1}{j^2}, j \in \mathbb{R}^{+}
\end{equation}
因此,在99\%的置信度下,排队时间的上界可以通过以下方式得到
\begin{equation}
	\sup_{\operatorname{Pr}}{\operatorname{q}_{d, v}^t} \leq \operatorname{\bar{q}}_{d, v}^t + 10  \sqrt{{Var}(\operatorname{q}_{d, v}^t)}
\end{equation}

为了更好地分析$D_v^t$中不同元素的平均排队时间和上传优先级之间的关系,公式\ref{equ 3-6}被改写为如下。
\begin{equation}
\overline{\mathrm{q}}_{d, v}^t=\frac{\rho_{d, v}^t \alpha_{d, v}^t}{1-\rho_{d, v}^t}+\frac{\lambda_{d, v}^t \beta_{d, v}^t+\sum_{\forall d^* \in D_{d, v}^t}^t \lambda_{d^*, v}^t \beta_{d^*, v}^t}{2\left(1-\rho_{d, v}^t\right)\left(1-\rho_{d, v}^t-\lambda_{d, v}^t \alpha_{d, v}^t\right)}
\end{equation}
假设有$n$种信息,信息${d^1}$具有最高的上传优先权,即$D_{d^1, v}^t = \emptyset$。
那么,信息${d^1}$的平均排队时间可以通过以下方式计算出来
\begin{equation}
\operatorname{\bar{q}}_{d^{1}, v}^t=\frac{\lambda_{d^1, v}^t \beta_{d^1, v}^t}{2}
\end{equation}
其中$\lambda_{d^1, v}^t$和$\beta_{d^1, v}^t$分别为信息$d^1$的感应频率和传输时间的二阶矩。
另一方面,信息${d^n}$的上传优先级最低。
由于要求$\rho_v^t < 1$以保证队列的稳定性和排队时间的有限性,我们有
\begin{equation}
\rho_{d, v}^t=\sum_{\forall d^* \in D_{d, v}^t} \lambda_{d^*, v}^t \alpha_{d^*, v}^t<\sum_{\forall d \subseteq D_v^t} \lambda_{d, v}^t \alpha_{d, v}^t=\rho_v^t<1
\end{equation}
同样地, $\rho_{d, v}^t+\lambda_{d, v}^t \alpha_{d, v}^t<1$。
当 $n$ 趋于无穷大时,由于$\lim _{n \rightarrow \infty}(1-\rho_{d^n, v}^t) \rightarrow 0$,类似地,$\lim _{n \rightarrow \infty}(1-\rho_{d^n, v}^t-\lambda_{d^n, v}^t \alpha_{d^n, v}^t) \rightarrow 0$,所以信息${d^n}$的平均排队时间可以得到
\begin{equation}
\begin{aligned}
	\lim _{n \rightarrow \infty}\left(\mathrm{\bar{q}}_{d^n, v}^t\right)&=\frac{\lambda_{d^n, v}^t \beta_{d^n, v}^t+\sum_{\forall d^* \in D_{d^n, v}^t} \lambda_{d^*, v}^t \beta_{d^*, v}^t}{2\left(1-\rho_{d^n, v}^t\right)\left(1-\rho_{d^n, v}^t-\lambda_{d^n, v}^t \alpha_{d^n, v}^t\right)} + \frac{\rho_{d^n, v^t}^t \alpha_{d^n, v}^t}{1-\rho_{d^n, v}^t}\rightarrow \infty
\end{aligned}
\end{equation}
其中,$\lambda_{d^n, v}^t$、$\alpha_{d^n, v}^t$和$\beta_{d^n, v}^t$分别为信息$d^n$的感应频率、传输时间的平均值和二阶矩。


\section{定理\ref{theorem 4-1}的证明}
\label{appendix f}
\begin{proof} 根据公式\ref{equ 4-16},可以得到
\begin{align}
&{F}_{e}\left(\mathcal{{S}}^{\prime}_{e}, \mathcal{S}_{-e}\right) - {F}_{e}\left(\mathcal{S}_{e}, \mathcal{S}_{-e}\right) \notag \\
		&={U}_{e}\left(\mathcal{S}^{\prime}_{e}, \mathcal{S}_{-e}\right) - {U}_{e}\left(-\mathcal{S}^{\prime}_{e}, \mathcal{S}_{-e}\right) - \left( {U}_{e}\left(\mathcal{S}_{e}, \mathcal{S}_{-e}\right) - {U}_{e}\left(-\mathcal{S}_{e}, \mathcal{S}_{-e}\right) \right) \notag \\
		&={U}_{e}\left(\mathcal{S}^{\prime}_{e}, \mathcal{S}_{-e}\right) - {U}_{e}\left(\mathcal{S}_{e}, \mathcal{S}_{-e}\right) + {U}_{e}\left(-\mathcal{S}_{e}, \mathcal{S}_{-e}\right) - {U}_{e}\left(-\mathcal{S}^{\prime}_{e}, \mathcal{S}_{-e}\right) \notag \\
		&={U}_{e}\left(\mathcal{S}^{\prime}_{e}, \mathcal{S}_{-e}\right) - {U}_{e}\left(\mathcal{S}_{e}, \mathcal{S}_{-e}\right)
\end{align}
因此,定理\ref{theorem 4-1}得证。
\end{proof}

\section{引理\ref{lemma 4-1}的证明}
\label{appendix g}
\begin{proof}
假定 $\mathcal{S}^{*}$ 是博弈 $\mathcal{G}$ 的一个纳什均衡,可以得到
\begin{equation}
	U_{e}\left(\mathcal{S}_{e}^{*}, \mathcal{S}_{-e}^{*}\right) - U_{e}\left(\mathcal{S}_{e}, \mathcal{S}_{-e}^{*}\right) \geq 0, \quad \forall \mathcal{S}_{e} \in \mathbf{S}_{e}, \forall e \in \mathbf{E}
\end{equation}
根据严格势博弈的定义,可以得到
\begin{equation}
	F_{e}\left(\mathcal{S}_{e}^{*}, \mathcal{S}_{-e}^{*}\right) - F_{e}\left(\mathcal{S}_{e}, \mathcal{S}_{-e}^{*}\right) \geq 0, \quad \forall \mathcal{S}_{e} \in \mathbf{S}_{e}, \forall e \in \mathbf{E}
\end{equation}
因此,$\mathcal{S}^{*}$ 也是博弈 $\mathcal{G}^{F}$ 的纳什均衡,且 $\operatorname{NE}(\mathcal{G}^{F}) \subseteq \operatorname{NE}(\mathcal{G})$。
同样地,$\operatorname{NE}(\mathcal{G}) \subseteq \operatorname{NE}(\mathcal{F})$。
\end{proof}

\section{定理\ref{theorem 4-2}的证明}
\label{appendix h}
\begin{proof}
策略空间 $\mathbb{S}$ 是封闭和有界的。 
因此,势函数 $F_{e}(\mathcal{S})$ 在空间 $\mathbb{S}$ 内至少有一个最大点,也就对应于博弈 $\mathcal{G}^{F}$ 的纳什均衡。 
然后,根据引理 \ref{lemma 4-1},博弈 $\mathcal{G}$ 至少有一个纯策略纳什均衡。
\end{proof}

\section{定理\ref{theorem 4-3}的证明}
\label{appendix i}
\begin{proof}
由于博弈$\mathcal{G}$的策略空间$\mathbb{S}$是封闭且有界的,$\exists F^{\max} \in \mathbb{R}$ 且 $F^{\max} < \infty$,可得$F^{\max} = \sup _{\mathcal{S} \in \mathbb{S}} F_{e}(S)$。
假定路径 $\rho=\left(\mathcal{S}^{0}, \mathcal{S}^{1}, \ldots, \mathcal{S}^{i}, \ldots\right)$ 是一条 $\epsilon$改进路径,且该路径是无限的。
基于$\epsilon$改进路径的定义,可得$U_{e}\left(\mathcal{S}^{i+1}\right) > U_{e}\left(\mathcal{S}^{i}\right) + \epsilon, \exists \epsilon \in \mathbb{R}_{+}, \forall i$。
因此,可得$F_{e}\left(\mathcal{S}^{i+1}\right) > F_{e}\left(\mathcal{S}^{i}\right) + \epsilon^{\prime}, \exists \epsilon^{\prime} \in \mathbb{R}_{+}, \forall i$,其中 $\epsilon^{\prime}$ 是一个足够小常量。
可以进一步推出
\begin{align}
	&F_{e}\left(\mathcal{S}^{i}\right) > F_{e}\left(\mathcal{S}^{0}\right) + i   \epsilon^{\prime}, \forall i \\
	&\lim _{i \rightarrow \infty} F\left(S^{i}\right) > \lim _{i \rightarrow \infty} \left \{ F_{e}\left(\mathcal{S}^{0}\right) + i   \epsilon^{\prime} \right\} =\infty
\end{align}
这与 $F^{\max} < \infty$ 是相互矛盾的,也就说明路径 $\rho$ 必须是有限的,并且终止于 $\epsilon$均衡点。
\end{proof}

\section{学位论文数据集}

\begin{table}[h]
\resizebox{\columnwidth}{!}{%
\begin{tabular}{|cllcclclcl|}
\hline
\multicolumn{4}{|c|}{\heiti{关键词}}             & \multicolumn{2}{c|}{\heiti{密级}}   & \multicolumn{4}{c|}{\heiti{中图分类号}}                                    \\ \hline
\multicolumn{4}{|c|}{\begin{tabular}[c]{@{}c@{}}车载信息物理融合系统;\\异构车联网; 车载边缘计算;\\资源优化; 多智能体深度强化学习\end{tabular}} & \multicolumn{2}{c|}{公开} & \multicolumn{4}{c|}{TP} \\ \hline
\multicolumn{3}{|c|}{\heiti{学位授予单位名称}} & \multicolumn{3}{c|}{\heiti{学位授予单位代码}}    & \multicolumn{2}{c|}{\heiti{学位类别}}  & \multicolumn{2}{c|}{\heiti{学位级别}}        \\ \hline
\multicolumn{3}{|c|}{\secretize{重庆大学}}     & \multicolumn{3}{c|}{\secretize{10611}}       & \multicolumn{2}{c|}{学术学位}  & \multicolumn{2}{c|}{博士}          \\ \hline
\multicolumn{4}{|c|}{\heiti{论文题名}}            & \multicolumn{2}{c|}{\heiti{并列题名}} & \multicolumn{4}{c|}{\heiti{论文语种}}                                     \\ \hline
\multicolumn{4}{|c|}{\begin{tabular}[c]{@{}c@{}}车载信息物理融合系统关键技术研究\end{tabular}}               & \multicolumn{2}{c|}{}   & \multicolumn{4}{c|}{中文} \\ \hline
\multicolumn{3}{|c|}{\heiti{作者姓名}}     & \multicolumn{3}{c|}{\secretize{许新操}}         & \multicolumn{2}{c|}{\heiti{学号}}    & \multicolumn{2}{c|}{\secretize{20191401452}} \\ \hline
\multicolumn{6}{|c|}{\heiti{培养单位名称}}                                      & \multicolumn{4}{c|}{\heiti{培养单位代码}}                                   \\ \hline
\multicolumn{6}{|c|}{\secretize{重庆大学}}                                        & \multicolumn{4}{c|}{\secretize{10611}}                                    \\ \hline
\multicolumn{3}{|c|}{\heiti{学科专业}}     & \multicolumn{3}{c|}{\heiti{研究方向}}        & \multicolumn{2}{c|}{\heiti{学制}}    & \multicolumn{2}{c|}{\heiti{学位授予年}}       \\ \hline
\multicolumn{3}{|c|}{计算机科学与技术} & \multicolumn{3}{c|}{车联网}         & \multicolumn{2}{c|}{4}     & \multicolumn{2}{c|}{\secretize{2023}}        \\ \hline
\multicolumn{3}{|c|}{\heiti{论文提交日期}}   & \multicolumn{3}{c|}{\secretize{2023年6月}}     & \multicolumn{2}{c|}{\heiti{论文总页数}} & \multicolumn{2}{c|}{\pageref{LastPage}}         \\ \hline
\multicolumn{3}{|c|}{\heiti{导师姓名}}     & \multicolumn{3}{c|}{\secretize{刘凯}}          & \multicolumn{2}{c|}{\heiti{职称}}    & \multicolumn{2}{c|}{教授}          \\ \hline
\multicolumn{6}{|c|}{\heiti{答辩委员会主席}}                                     & \multicolumn{4}{c|}{\secretize{雒江涛}}                                      \\ \hline
\multicolumn{10}{|l|}{\heiti{\begin{tabular}[c]{@{}l@{}}电子版论文提交格式\\ 文本(\checkmark) 图像() 视频()音频()多媒体()其他()\end{tabular}}}                              \\ \hline
\end{tabular}%
}
\end{table}
