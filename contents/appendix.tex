\chapter[附\hskip\ccwd{}\hskip\ccwd{}录]{{\CJKfontspec{SimHei}\zihao{3}附\hskip\ccwd{}\hskip\ccwd{}录}}
\section[\hspace{-2pt}作者在攻读学位期间的论文目录]{{\CJKfontspec{SimHei}\zihao{-3} \hspace{-8pt}作者在攻读学位期间的论文目录}}

%下面是盲审标记\cs{secretize}的用法,记得去\textsf{main.tex}开启盲审开关看效果:

\circled{1}已发表论文

\begin{enumerate}
	\item \textbf{\secretize{XU X}}, \secretize{LIU K}, DAI P, et al. Joint task offloading and resource optimization in NOMA-based vehicular edge computing: A game-theoretic DRL approach[J]. Journal of Systems Architecture, 2023, 134: 102780. 影响因子: 5.836(2021), 4.497(5年) (中科院SCI 2区,对应本文第三章)
	\item \textbf{\secretize{许新操}}, \secretize{刘凯}, 刘春晖, 等. 基于势博弈的车载边缘计算信道分配方法[J]. 电子学报, 2021,49(5): 851-860. (EI 索引,CCF T1类中文高质量科技期刊,对应本文第三章)
	\item \textbf{ \secretize{XU X}}, \secretize{LIU K}, XIAO K, et al. Vehicular fog computing enabled real-time collision warning via trajectory calibration[J]. Mobile Networks and Applications, 2020, 25(6): 2482-2494. 影响因子: 3.077(2021), 2.92(5年) (中科院SCI 3区,对应本文第五章)
	\item \secretize{LIU K}, \textbf{\secretize{XU X}}, CHEN M, et al. A hierarchical architecture for the future Internet of Vehicles[J]. IEEE Communications Magazine, 2019, 57(7): 41-47. 影响因子: 9.03(2021), 10.892(5年) (中科院SCI 1区,对应本文第二章)
	\item \textbf{ \secretize{XU X}}, \secretize{LIU K}, ZHANG Q, et al. Age of view: A new metric for evaluating heterogeneous information fusion in vehicular cyber-physical systems[C]. Proceedings of IEEE International Conference on Intelligent Transportation Systems (IEEE ITSC’22), Macau, China, October 8-12, 2022. (EI 索引)
	\item \textbf{\secretize{许新操}}, 周易, \secretize{刘凯}, 等. 车载雾计算环境中基于势博弈的分布式信道分配[C]. 第十四届中国物联网学术会议(CWSN’20), 中国敦煌, 2020/9/18-9/21.
	\item \textbf{\secretize{XU X}}, \secretize{LIU K}, XIAO K, et al. Design and implementation of a fog computing based collision warning system in VANETs[C]. Proceedings of IEEE International Symposium on Product Compliance Engineering-Asia (IEEE ISPCE-CN’18), Hong Kong/Shengzhen, December 5-7, 2018. (EI 索引)
	\item LIU C, \secretize{LIU K}, REN H, \textbf{\secretize{XU X}}, et al. RtDS: Real-time distributed strategy for multi-period task offloading in vehicular edge computing environment[J]. Neural Computing and Applications, to appear. 影响因子: 5.102(2021), 5.13(5年) (中科院SCI 2区)
	\item XIAO K, \secretize{LIU K}, \textbf{\secretize{XU X}}, et al. Cooperative coding and caching scheduling via binary particle swarm optimization in software defined vehicular networks[J]. Neural Computing and Applications, 2021, 33(5): 1467-1478. 影响因子: 5.102(2021), 5.13(5年) (中科院SCI 2区)
	\item XIAO K, \secretize{LIU K}, \textbf{\secretize{XU X}}, et al. Efficient fog-assisted heterogeneous data services in software defined VANETs[J]. Journal of Ambient Intelligence and Humanized Computing, 2021, 12(1): 261-273. 影响因子: 3.662 (2021), 3.718 (5年) (中科院SCI 3区)
	\item LIU C, \secretize{LIU K}, \textbf{\secretize{XU X}}, et al. Real-time task offloading for data and computation intensive services in vehicular fog computing environments[C]. Proceedings of IEEE International Conference on Mobility, Sensing and Networking (IEEE MSN’20), Tokyo, Japan, December 17-19, 2020. (EI 索引,CCF C类国际会议)
	\item ZHOU Y, \secretize{LIU K}, \textbf{ \secretize{XU X}}, et al. Multi-period distributed delay-sensitive tasks offloading in a two-layer vehicular fog computing architecture[C]. Proceedings of International Conference on Neural Computing and Applications (NCAA’20), Shenzhen, China, July 3-6, 2020. (EI 索引)
	\item ZHOU Y, \secretize{LIU K}, \textbf{ \secretize{XU X}}, et al. Distributed scheduling for time-critical tasks in a two-layer vehicular fog computing architecture[C]. Proceedings of IEEE Consumer Communications and Networking Conference (IEEE CCNC’20), Las Vegas, USA, January 11-14, 2020. (EI 索引)
\end{enumerate}

\circled{2}已投稿论文

\begin{enumerate}
	\item \textbf{\secretize{XU X}}, \secretize{LIU K}, DAI P, et al. Cooperative sensing and heterogeneous information fusion in VCPS: A multi-agent deep reinforcement learning approach[J]. IEEE Transactions on Intelligent Transportation Systems, under major revision. 影响因子: 9.551 (2021), 9.502 (5年) (中科院SCI 1区,对应本文第二章)
	\item \secretize{LIU K},\textbf{\secretize{XU X}}, DAI P, et al. Cooperative sensing and uploading for quality-cost tradeoff of digital twins in VEC[J]. IEEE Transactions on Consumer Electronics, under minor revision. 影响因子: 4.414 (2021), 3.565 (5年) (中科院SCI 2区,对应本文第四章) 
\end{enumerate}

\section[\hspace{-2pt}作者在攻读学位期间取得的科研成果目录]{{\CJKfontspec{SimHei}\zihao{-3} \hspace{-8pt}作者在攻读学位期间取得的科研成果目录}}
\begin{enumerate}
	\item \textbf{\secretize{许新操}}, \secretize{刘凯}, 李东. 一种针对软件定义车联网的控制平面视图构建方法. 发明专利. ZL202110591822.1.
	\item \secretize{刘凯}, 张浪, \textbf{\secretize{许新操}}, 任华玲, 周易. 一种基于边缘计算的盲区车辆碰撞预警方法. 发明专利. ZL201910418745.2.
	\item 任华玲, \secretize{刘凯}, 陈梦良, 周易, \textbf{\secretize{许新操}}. 一种基于雾计算的信息采集、计算、传输架构. 发明专利. ZL201910146357.3.
\end{enumerate}

\section[\hspace{-2pt}作者在攻读学位期间参与的科研项目目录]{{\CJKfontspec{SimHei}\zihao{-3} \hspace{-8pt}作者在攻读学位期间参与的科研项目目录}}
\begin{enumerate}
	\item 国家自然科学基金面上项目,面向车联网边缘智能的计算模型部署与协同跨域优化,项目编号: 62172064,2022/01–2025/12.(项目参与人员)
	\item 国家自然科学基金面上项目,面向大规模数据服务的异构融合车联网架构与协议研究,项目编号: 61872049,2019/01–2022/12.(项目参与人员)
\end{enumerate}

\section[\hspace{-2pt}学位论文相关代码]{{\CJKfontspec{SimHei}\zihao{-3} \hspace{-8pt}学位论文相关代码}}
\begin{enumerate}
	\item 基于差分奖励的多智能体深度强化学习源代码\\https://github.com/neardws/Multi-Agent-Deep-Reinforcement-Learning
	\item 基于博弈理论的多智能体深度强化学习源代码\\https://github.com/neardws/Game-Theoretic-Deep-Reinforcement-Learning
	\item 基于多目标的多智能体深度强化学习源代码\\https://github.com/neardws/MAMO-Deep-Reinforcement-Learning
	\item 基于车载信息物理融合系统优化的碰撞预警源代码\\https://github.com/neardws/fog-computing-based-collision-warning-system
	\item 基于C-V2X通信的碰撞预警原型系统源代码\\https://github.com/neardws/V2X-based-Collision-Warning
	\item 基于DSRC通信的碰撞预警原型系统源代码\\https://github.com/cqu-bdsc/Collision-Warning-System
	\item 滴滴 GAIA 数据集处理源代码\\https://github.com/neardws/Vehicular-Trajectories-Processing-for-Didi-Open-Data
\end{enumerate}

\newpage
\section[\hspace{-2pt}学位论文数据集]{{\CJKfontspec{SimHei}\zihao{-3} \hspace{-8pt}学位论文数据集}}

\begin{table}[h]
\resizebox{\columnwidth}{!}{%
\begin{tabular}{|cllcclclcl|}
\hline
\multicolumn{4}{|c|}{\heiti{关键词}}             & \multicolumn{2}{c|}{\heiti{密级}}   & \multicolumn{4}{c|}{\heiti{中图分类号}}                                    \\ \hline
\multicolumn{4}{|c|}{\begin{tabular}[c]{@{}c@{}}车载信息物理融合系统;\\异构车联网; 车载边缘计算;\\资源优化; 多智能体深度强化学习\end{tabular}} & \multicolumn{2}{c|}{公开} & \multicolumn{4}{c|}{TP} \\ \hline
\multicolumn{3}{|c|}{\heiti{学位授予单位名称}} & \multicolumn{3}{c|}{\heiti{学位授予单位代码}}    & \multicolumn{2}{c|}{\heiti{学位类别}}  & \multicolumn{2}{c|}{\heiti{学位级别}}        \\ \hline
\multicolumn{3}{|c|}{\secretize{重庆大学}}     & \multicolumn{3}{c|}{\secretize{10611}}       & \multicolumn{2}{c|}{学术学位}  & \multicolumn{2}{c|}{博士}          \\ \hline
\multicolumn{4}{|c|}{\heiti{论文题名}}            & \multicolumn{2}{c|}{\heiti{并列题名}} & \multicolumn{4}{c|}{\heiti{论文语种}}                                     \\ \hline
\multicolumn{4}{|c|}{\begin{tabular}[c]{@{}c@{}}车载信息物理融合系统建模与优化关键技术研究\end{tabular}}               & \multicolumn{2}{c|}{无}   & \multicolumn{4}{c|}{中文} \\ \hline
\multicolumn{3}{|c|}{\heiti{作者姓名}}     & \multicolumn{3}{c|}{\secretize{许新操}}         & \multicolumn{2}{c|}{\heiti{学号}}    & \multicolumn{2}{c|}{\secretize{20191401452}} \\ \hline
\multicolumn{6}{|c|}{\heiti{培养单位名称}}                                      & \multicolumn{4}{c|}{\heiti{培养单位代码}}                                   \\ \hline
\multicolumn{6}{|c|}{\secretize{重庆大学}}                                        & \multicolumn{4}{c|}{\secretize{10611}}                                    \\ \hline
\multicolumn{3}{|c|}{\heiti{学科专业}}     & \multicolumn{3}{c|}{\heiti{研究方向}}        & \multicolumn{2}{c|}{\heiti{学制}}    & \multicolumn{2}{c|}{\heiti{学位授予年}}       \\ \hline
\multicolumn{3}{|c|}{计算机科学与技术} & \multicolumn{3}{c|}{车联网}         & \multicolumn{2}{c|}{4年}     & \multicolumn{2}{c|}{\secretize{2023年}}        \\ \hline
\multicolumn{3}{|c|}{\heiti{论文提交日期}}   & \multicolumn{3}{c|}{\secretize{2023年6月}}     & \multicolumn{2}{c|}{\heiti{论文总页数}} & \multicolumn{2}{c|}{\pageref{LastPage}}         \\ \hline
\multicolumn{3}{|c|}{\heiti{导师姓名}}     & \multicolumn{3}{c|}{\secretize{刘凯}}          & \multicolumn{2}{c|}{\heiti{职称}}    & \multicolumn{2}{c|}{教授}          \\ \hline
\multicolumn{6}{|c|}{\heiti{答辩委员会主席}}                                     & \multicolumn{4}{c|}{\secretize{雒江涛}}                                      \\ \hline
\multicolumn{10}{|c|}{\heiti{\begin{tabular}[c]{@{}c@{}} 电子版论文提交格式\\ 文本(\checkmark) 图像() 视频()音频()多媒体()其他()\end{tabular}}}                              \\ \hline
\end{tabular}%
}
\end{table}