\chapter{附\hskip\ccwd{}\hskip\ccwd{}录}
\section{作者在攻读学位期间的论文目录}

%下面是盲审标记\cs{secretize}的用法,记得去\textsf{main.tex}开启盲审开关看效果:

\circled{1}已发表论文

\begin{enumerate}
	\item 作者\textbf{\secretize{Xincao Xu}}, 导师\secretize{Kai Liu}, Penglin Dai, Feiyu Jin, Hualing Ren, Choujun Zhan and Songtao Guo. Joint Task Offloading and Resource Optimization in NOMA-Based Vehicular Edge Computing: A Game-Theoretic DRL Approach[J]. Journal of Systems Architecture, 2023, 134: 102780. 影响因子: 5.836(2021), 4.497(5年) (中科院SCI 2区,对应本文第三章)
	\item 作者\textbf{\secretize{许新操}}, 导师\secretize{刘凯}, 刘春晖, 蒋豪, 郭松涛, 吴巍炜. 基于势博弈的车载边缘计算信道分配方法[J]. 电子学报, 2021,49(5): 851-860. (EI 索引,CCF T1类中文高质量科技期刊,对应本文第三章)
	\item 作者\textbf{\secretize{Xincao Xu}}, 导师\secretize{Kai Liu}, Ke Xiao, Liang Feng, Zhou Wu and Songtao Guo. Vehicular Fog Computing Enabled Real-time Collision Warning via Trajectory Calibration[J]. Mobile Networks and Applications, 2020, 25(6): 2482-2494. 影响因子: 3.077(2021), 2.92(5年) (中科院SCI 3区,对应本文第五章)
	\item 导师\secretize{Kai Liu}, 作者\textbf{\secretize{Xincao Xu}}, Mengliang Chen, Bingyi Liu, Libing Wu and Victor Lee. A Hierarchical Architecture for the Future Internet of Vehicles[J]. IEEE Communications Magazine, 2019, 57(7): 41-47. 影响因子: 9.03(2021), 10.892(5年) (中科院SCI 1区,对应本文第二章)
	\item 作者\textbf{\secretize{Xincao Xu}}, 导师\secretize{Kai Liu}, Qisen Zhang, Hao Jiang, Ke Xiao and Jiangtao Luo. Age of View: A New Metric for Evaluating Heterogeneous Information Fusion in Vehicular Cyber-Physical Systems[C]. Proceedings of IEEE International Conference on Intelligent Transportation Systems (IEEE ITSC’22), Macau, China, October 8-12, 2022. (EI 索引)
	\item 作者\textbf{\secretize{许新操}}, 周易, 导师\secretize{刘凯}, 向朝参, 李艳涛, 郭松涛. 车载雾计算环境中基于势博弈的分布式信道分配[C]. 第十四届中国物联网学术会议(CWSN’20), 中国敦煌, 2020/9/18-9/21.
	\item 作者\textbf{\secretize{Xincao Xu}}, 导师\secretize{Kai Liu}, Ke Xiao, Hualing Ren, Liang Feng and Chao Chen. Design and Implementation of a Fog Computing Based Collision Warning System in VANETs[C]. Proceedings of IEEE International Symposium on Product Compliance Engineering-Asia (IEEE ISPCE-CN’18), Hong Kong/Shengzhen, December 5-7, 2018. (EI 索引)
	\item Chunhui Liu, 导师\secretize{Kai Liu}, Hualing Ren, 作者\textbf{\secretize{Xincao Xu}}, Ruitao Xie and Jingjing Cao. RtDS: Real-time Distributed Strategy for Multi-period Task Offloading in Vehicular Edge Computing Environment[J]. Neural Computing and Applications, to appear. 影响因子: 5.102(2021), 5.13(5年) (中科院SCI 2区)
	\item Ke Xiao, 导师\secretize{Kai Liu}, 作者\textbf{\secretize{Xincao Xu}}, Liang Feng, Zhou Wu and Qiangwei Zhao. Cooperative Coding and Caching Scheduling via Binary Particle Swarm Optimization in Software Defined Vehicular Networks[J]. Neural Computing and Applications, 2021, 33(5): 1467-1478. 影响因子: 5.102(2021), 5.13(5年) (中科院SCI 2区)
	\item Ke Xiao, 导师\secretize{Kai Liu}, 作者\textbf{\secretize{Xincao Xu}}, Yi Zhou and Liang Feng. Efficient Fog-assisted Heterogeneous Data Services in Software Defined VANETs[J]. Journal of Ambient Intelligence and Humanized Computing, 2021, 12(1): 261-273. 影响因子: 3.662 (2021), 3.718 (5年) (中科院SCI 3区)
	\item Chunhui Liu, 导师\secretize{Kai Liu}, 作者\textbf{\secretize{Xincao Xu}}, Hualing Ren, Feiyu Jin and Songtao Guo. Real-time Task Offloading for Data and Computation Intensive Services in Vehicular Fog Computing Environments[C]. Proceedings of IEEE International Conference on Mobility, Sensing and Networking (IEEE MSN’20), Tokyo, Japan, December 17-19, 2020. (EI 索引,CCF C类国际会议)
	\item Yi Zhou, 导师\secretize{Kai Liu}, 作者\textbf{\secretize{Xincao Xu}}, Chunhui Liu, Liang Feng and Chao Chen. Multi-period Distributed Delay-sensitive Tasks Offloading in a Two-layer Vehicular Fog Computing Architecture[C]. Proceedings of International Conference on Neural Computing and Applications (NCAA’20), Shenzhen, China, July 3-6, 2020. (EI 索引)
	\item Yi Zhou, 导师\secretize{Kai Liu}, 作者\textbf{\secretize{Xincao Xu}}, Songtao Guo, Zhou Wu, Victor Lee and Sang Son. Distributed Scheduling for Time-Critical Tasks in a Two-layer Vehicular Fog Computing Architecture[C]. Proceedings of IEEE Consumer Communications and Networking Conference (IEEE CCNC’20), Las Vegas, USA, January 11-14, 2020. (EI 索引)
\end{enumerate}

\circled{2}已投稿论文

\begin{enumerate}
	\item 作者\textbf{\secretize{Xincao Xu}}, 导师\secretize{Kai Liu}, Penglin Dai, Ruitao Xie, and Jiangtao Luo. Cooperative Sensing and Heterogeneous Information Fusion in VCPS: A Multi-agent Deep Reinforcement Learning Approach [J]. IEEE Transactions on Intelligent Transportation Systems, Major Revisions. 影响因子: 9.551 (2021), 9.502 (5年) (中科院SCI 1区,对应本文第二章)
	\item 导师\secretize{Kai Liu},作者\textbf{\secretize{Xincao Xu}}, Penglin Dai, and Biwen Chen. Cooperative Sensing and Uploading for Quality-Cost Tradeoff of Digital Twins in VEC [J]. IEEE Transactions on Consumer Electronics, Major Revisions. 影响因子: 4.414 (2021), 3.565 (5年) (中科院SCI 2区,对应本文第四章) 
\end{enumerate}

\section{作者在攻读学位期间取得的科研成果目录}
\begin{enumerate}
	\item 作者\textbf{\secretize{许新操}}, 导师\secretize{刘凯}, 李东. 一种针对软件定义车联网的控制平面视图构建方法. 发明专利. ZL202110591822.1.
	\item 导师\secretize{刘凯}, 张浪, 作者\textbf{\secretize{许新操}}, 任华玲, 周易. 一种基于边缘计算的盲区车辆碰撞预警方法. 发明专利. ZL201910418745.2.
	\item 任华玲, 导师\secretize{刘凯}, 陈梦良, 周易, 作者\textbf{\secretize{许新操}}. 一种基于雾计算的信息采集、计算、传输架构. 发明专利. ZL201910146357.3.
\end{enumerate}

\section{作者在攻读学位期间参与的科研项目目录}
\begin{enumerate}
	\item 国家自然科学基金面上项目,“面向车联网边缘智能的计算模型部署与协同跨域优化”,项目编号: 62172064,2022/01–2025/12.(项目参与人员)
	\item 国家自然科学基金面上项目,“面向大规模数据服务的异构融合车联网架构与协议研究”,项目编号: 61872049,2019/01–2022/12.(项目参与人员)
\end{enumerate}

\section{学位论文相关代码}
\begin{enumerate}
	\item 滴滴 GAIA 数据集处理源代码\\https://github.com/neardws/Vehicular-Trajectories-Processing-for-Didi-Open-Data
	\item 基于差分奖励的多智能体强化学习源代码\\https://github.com/neardws/Multi-Agent-Deep-Reinforcement-Learning
	\item 基于博弈理论的多智能体强化学习源代码\\https://github.com/neardws/Game-Theoretic-Deep-Reinforcement-Learning
	\item 基于多目标的多智能体强化学习源代码\\https://github.com/neardws/MAMO-Deep-Reinforcement-Learning
	\item 基于视图校准的碰撞预警源代码\\https://github.com/neardws/fog-computing-based-collision-warning-system
	\item 基于C-V2X通信的碰撞预警原型系统实现源代码\\https://github.com/neardws/V2X-based-Collision-Warning
\end{enumerate}

\newpage
\section{学位论文数据集}

% 盲审格式空表
\begin{table}[h]
\resizebox{\columnwidth}{!}{%
\begin{tabular}{|cllcclclcl|}
\hline
\multicolumn{4}{|c|}{\heiti{关键词}}         & \multicolumn{2}{c|}{\heiti{密级}}   & \multicolumn{4}{c|}{\heiti{中图分类号}}                      \\ \hline
\multicolumn{4}{|c|}{}            & \multicolumn{2}{c|}{}     & \multicolumn{4}{c|}{}                           \\ \hline
\multicolumn{3}{|c|}{\heiti{学位授予单位名称}} & \multicolumn{3}{c|}{\heiti{学位授予单位代码}} & \multicolumn{2}{c|}{\heiti{学位类别}}  & \multicolumn{2}{c|}{\heiti{学位级别}}  \\ \hline
\multicolumn{3}{|c|}{}     & \multicolumn{3}{c|}{}            & \multicolumn{2}{c|}{}   & \multicolumn{2}{c|}{} \\ \hline
\multicolumn{4}{|c|}{\heiti{论文题名}}        & \multicolumn{2}{c|}{\heiti{并列题名}} & \multicolumn{4}{c|}{\heiti{论文语种}}                       \\ \hline
\multicolumn{4}{|c|}{}            & \multicolumn{2}{c|}{}     & \multicolumn{4}{c|}{}                           \\ \hline
\multicolumn{3}{|c|}{\heiti{作者姓名}} & \multicolumn{3}{c|}{}            & \multicolumn{2}{c|}{\heiti{学号}} & \multicolumn{2}{c|}{} \\ \hline
\multicolumn{6}{|c|}{\heiti{培养单位名称}}                                  & \multicolumn{4}{c|}{\heiti{培养单位代码}}                     \\ \hline
\multicolumn{6}{|c|}{}                                        & \multicolumn{4}{c|}{}                           \\ \hline
\multicolumn{3}{|c|}{\heiti{学科专业}}     & \multicolumn{3}{c|}{\heiti{研究方向}}     & \multicolumn{2}{c|}{\heiti{学制}}    & \multicolumn{2}{c|}{\heiti{学位授予年}} \\ \hline
\multicolumn{3}{|c|}{}     & \multicolumn{3}{c|}{}            & \multicolumn{2}{c|}{}   & \multicolumn{2}{c|}{} \\ \hline
\multicolumn{3}{|c|}{\heiti{论文提交日期}}   & \multicolumn{3}{c|}{}         & \multicolumn{2}{c|}{\heiti{论文总页数}} & \multicolumn{2}{c|}{}      \\ \hline
\multicolumn{3}{|c|}{\heiti{导师姓名}} & \multicolumn{3}{c|}{}            & \multicolumn{2}{c|}{\heiti{职称}} & \multicolumn{2}{c|}{} \\ \hline
\multicolumn{6}{|c|}{\heiti{答辩委员会主席}}                                 & \multicolumn{4}{c|}{}                           \\ \hline
\multicolumn{10}{|l|}{\begin{tabular}[c]{@{}l@{}}\heiti{电子版论文提交格式}\\ \heiti{文本()图像()视频()音频()多媒体()其他()}\end{tabular}}            \\ \hline
\end{tabular}%
}
\end{table}

%
%\begin{table}[h]
%\resizebox{\columnwidth}{!}{%
%\begin{tabular}{|cllcclclcl|}
%\hline
%\multicolumn{4}{|c|}{\heiti{关键词}}             & \multicolumn{2}{c|}{\heiti{密级}}   & \multicolumn{4}{c|}{\heiti{中图分类号}}                                    \\ \hline
%\multicolumn{4}{|c|}{\begin{tabular}[c]{@{}c@{}}车载信息物理融合系统;\\异构车联网; 车载边缘计算;\\资源优化; 多智能体深度强化学习\end{tabular}} & \multicolumn{2}{c|}{公开} & \multicolumn{4}{c|}{TP} \\ \hline
%\multicolumn{3}{|c|}{\heiti{学位授予单位名称}} & \multicolumn{3}{c|}{\heiti{学位授予单位代码}}    & \multicolumn{2}{c|}{\heiti{学位类别}}  & \multicolumn{2}{c|}{\heiti{学位级别}}        \\ \hline
%\multicolumn{3}{|c|}{\secretize{重庆大学}}     & \multicolumn{3}{c|}{\secretize{10611}}       & \multicolumn{2}{c|}{学术学位}  & \multicolumn{2}{c|}{博士}          \\ \hline
%\multicolumn{4}{|c|}{\heiti{论文题名}}            & \multicolumn{2}{c|}{\heiti{并列题名}} & \multicolumn{4}{c|}{\heiti{论文语种}}                                     \\ \hline
%\multicolumn{4}{|c|}{\begin{tabular}[c]{@{}c@{}}车载信息物理融合系统关键技术研究\end{tabular}}               & \multicolumn{2}{c|}{}   & \multicolumn{4}{c|}{中文} \\ \hline
%\multicolumn{3}{|c|}{\heiti{作者姓名}}     & \multicolumn{3}{c|}{\secretize{许新操}}         & \multicolumn{2}{c|}{\heiti{学号}}    & \multicolumn{2}{c|}{\secretize{20191401452}} \\ \hline
%\multicolumn{6}{|c|}{\heiti{培养单位名称}}                                      & \multicolumn{4}{c|}{\heiti{培养单位代码}}                                   \\ \hline
%\multicolumn{6}{|c|}{\secretize{重庆大学}}                                        & \multicolumn{4}{c|}{\secretize{10611}}                                    \\ \hline
%\multicolumn{3}{|c|}{\heiti{学科专业}}     & \multicolumn{3}{c|}{\heiti{研究方向}}        & \multicolumn{2}{c|}{\heiti{学制}}    & \multicolumn{2}{c|}{\heiti{学位授予年}}       \\ \hline
%\multicolumn{3}{|c|}{计算机科学与技术} & \multicolumn{3}{c|}{车联网}         & \multicolumn{2}{c|}{4}     & \multicolumn{2}{c|}{\secretize{2023}}        \\ \hline
%\multicolumn{3}{|c|}{\heiti{论文提交日期}}   & \multicolumn{3}{c|}{\secretize{2023年6月}}     & \multicolumn{2}{c|}{\heiti{论文总页数}} & \multicolumn{2}{c|}{\pageref{LastPage}}         \\ \hline
%\multicolumn{3}{|c|}{\heiti{导师姓名}}     & \multicolumn{3}{c|}{\secretize{刘凯}}          & \multicolumn{2}{c|}{\heiti{职称}}    & \multicolumn{2}{c|}{教授}          \\ \hline
%\multicolumn{6}{|c|}{\heiti{答辩委员会主席}}                                     & \multicolumn{4}{c|}{\secretize{雒江涛}}                                      \\ \hline
%\multicolumn{10}{|l|}{\heiti{\begin{tabular}[c]{@{}l@{}}电子版论文提交格式\\ 文本(\checkmark) 图像() 视频()音频()多媒体()其他()\end{tabular}}}                              \\ \hline
%\end{tabular}%
%}
%\end{table}
