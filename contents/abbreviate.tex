% !TeX encoding = UTF-8
% 环境用两个长度参数,分别定义左边距以及词条和解释的水平距离,可自己调试以达美观(全去掉时默认:20mm,30mm)

\begin{abbreviate}[0mm][18mm]
\item[3GPP] The 3rd Generation Partnership Project\hspace{1em}第三代合作伙伴计划
\item[5G] The 5th Generation Mobile Communication\hspace{1em}第五代移动通信
\item[5GAA] 5G Automotive Association\hspace{1em}5G汽车协会
\item[AA] Action-Advantage\hspace{1em}动作优势
\item[AAP] Average Achieved Potential\hspace{1em}平均实现势
\item[Adam] Adaptive Moment Estimation\hspace{1em}自适应矩估计
\item[ADMM] Alternating Direction Method of Multipliers\hspace{1em}交替方向乘子法
\item[AI] Artificial Intelligence\hspace{1em}人工智能
\item[AP] Access Point\hspace{1em}接入点
\item[API] Application Programming Interface\hspace{1em}应用程序编程接口
\item[APT] Average Processing Time\hspace{1em}平均处理时间
\item[AQT] Average Queuing Time\hspace{1em}平均排队时间
\item[AR] Average Redundancy\hspace{1em}平均冗余度
\item[ASC] Average Sensing Cost\hspace{1em}平均感知开销
\item[ASR] Average Service Ratio\hspace{1em}平均服务率
\item[AST] Average Service Time\hspace{1em}平均服务时间
\item[AT] Average Timeliness\hspace{1em}平均时效性
\item[ATC] Average Transmission Cost\hspace{1em}平均传输开销
\item[AWGN] Additive White Gaussian Noise\hspace{1em}加性白高斯噪声
\item[C-V2X] Cellular Vehicle-to-Everything\hspace{1em}蜂窝车联网
\item[CA] Collision Avoidance\hspace{1em}冲突避免
\item[CAR] Composition of Average Reward\hspace{1em}平均奖励构成
\item[CCW] Cloud-Based Collision Warning\hspace{1em}基于云的碰撞预警
\item[CP] Cyclic Prefix\hspace{1em}循环前缀
\item[CPS] Cyber-Physical System\hspace{1em}信息物理融合系统
\item[CR] Cumulative Reward\hspace{1em}累积奖励
\item[CRO] Cooperative Resource Optimization\hspace{1em}协同资源优化
\item[CSMA] Carrier-Sense Multiple Access\hspace{1em}载波侦听多路访问
\item[D4PG] Distributed Distributional Deep Deterministic Policy Gradient\hspace{1em}分布式深度确定性策略梯度
\item[DCN] Dueling Critic Network\hspace{1em}决斗评论家网络
\item[DDPG] Deep Deterministic Policy Gradient\hspace{1em}深度确定性策略梯度
\item[DoS] Denial-of-Service\hspace{1em}拒绝服务
\item[DP] Dynamic Programming\hspace{1em}动态规划
\item[DQN] Deep Q Networks\hspace{1em}深度 Q 网络
\item[DR] Difference Reward\hspace{1em}差分奖励
\item[DRL] Deep Reinforcement Learning\hspace{1em}深度强化学习
\item[DSRC] Dedicated Short-Range Communication\hspace{1em}专用短距通信
\item[ECW] Edge-Based Collision Warning\hspace{1em}基于边缘的碰撞预警
\item[EM] Expectation-Maximization\hspace{1em}期望最大化
\item[eMBB] enhanced Mobile Broadband\hspace{1em}增强型移动宽带
\item[EPG] Exact Potential Game\hspace{1em}严格势博弈
\item[FDI] False Data Injection\hspace{1em}虚假数据注入
\item[GNSS] Global Navigation Satellite System\hspace{1em}全球导航卫星系统
\item[GPS] Global Positioning System\hspace{1em}全球定位系统
\item[HARQ] Hybrid Automatic Repeat reQuest\hspace{1em}混合自动重传请求
\item[ICV] Intelligent Connected Vehicle\hspace{1em}智能网联汽车
\item[IEEE] Institute of Electrical and Electronics Engineers\hspace{1em}电气和电子工程师协会
\item[IoT] Internet of Things\hspace{1em}物联网
\item[IRS] Intelligent Reflecting Surface\hspace{1em}智能反射面
\item[ITS] Intelligent Transportation System\hspace{1em}智能交通系统
\item[KKT] Karush-Kuhn-Tucher\hspace{1em}卡罗需-库恩-塔克
\item[LDPC] Low Density Parity Check\hspace{1em}低密度奇偶校验
\item[LOS] Line-of-Sight\hspace{1em}视距
\item[LTE] Long-Term Evolution\hspace{1em}长期演进
\item[MAAC] Multi-Agent Actor-Critic\hspace{1em}多智能体行动者-评论家
\item[MAC] Media Access Control\hspace{1em}媒体接入
\item[MAD4PG] Multi-Agent D4PG\hspace{1em}多智能体分布式深度确定性策略梯度
\item[MADDPG] Multi-Agent DDPG\hspace{1em}多智能体深度确定性策略梯度
\item[MADR] Multi-Agent Difference-Reward-based DRL\hspace{1em}基于差分奖励的多智能体深度强化学习
\item[MADRL] Multi-Agent Deep Reinforcement Learning\hspace{1em}多智能体深度强化学习
\item[MAGT] Multi-Agent Game-Theoretic DRL\hspace{1em}基于博弈理论的多智能体强化学习
\item[MAMO] Multi-Agent Multi-Objective DRL\hspace{1em}基于多目标的多智能体深度强化学习
\item[MDR] Markov Decision Process\hspace{1em}马尔可夫决策过程
\item[MEC] Mobile Edge Computing\hspace{1em}移动边缘计算
\item[NE] Nash Equilibrium\hspace{1em}纳什均衡
\item[NFV] Network Functions Virtualization\hspace{1em}网络功能虚拟化
\item[NGA] Non-Cooperative Game\hspace{1em}非合作博弈
\item[NGMN] Next Generation Mobile Network\hspace{1em}下一代移动通信网络
\item[NLOS] Non-Line-of-Sight\hspace{1em}非视距
\item[NOMA] Non-Orthogonal Multiple Access\hspace{1em}非正交多址接入
\item[NR] New Radio\hspace{1em}新空口
\item[NS] Network Slicing\hspace{1em}网络切片
\item[OBU] Onboard Unit\hspace{1em}车载终端
\item[OFDM] Orthogonal Frequency Division Multiplexing\hspace{1em}正交频分复用
\item[OMA] Orthogonal Multiple Access\hspace{1em}正交多址接入
\item[ORL] Optimal Resource Allocation and Task Local Processing Only\hspace{1em}最优资源分配和任务仅本地处理
\item[ORM] Optimal Resource Allocation and Task Migration Only\hspace{1em}最优资源分配和任务全迁移
\item[OTA] Over-the-Air\hspace{1em}空中下载
\item[PDF] Probability Density Function\hspace{1em}概率密度函数
\item[PLPM] Proportion of Locally Processing to Migration\hspace{1em}本地处理与迁移的比例
\item[PPO] Proximal Policy Optimization\hspace{1em}近似策略优化
\item[PPUQ] Profit Per Unit Quality\hspace{1em}单位质量利润
\item[PSO] Particle Swarm Optimization\hspace{1em}粒子群优化
\item[QAM] Quadrature Amplitude Modulation\hspace{1em}正交幅度调制
\item[QoS] Quality of Service\hspace{1em}服务质量
\item[QPUC] Quality Per Unit Cost\hspace{1em}单位开销质量
\item[RA] Random Allocation\hspace{1em}随机分配
\item[ReLU] Rectified Linear Unit\hspace{1em}整流线性单元
\item[RSU] Roadside Unit\hspace{1em}路侧设备
\item[SC] Superposition Coding\hspace{1em}叠加编码
\item[SC-FDM] Single-Carrier Frequency-Division Multiplexing\hspace{1em}单载波频分复用
\item[SDN] Software Defined Network\hspace{1em}软件定义网络
\item[SDVN] Software Defined Vehicular Network\hspace{1em}软件定义车联网
\item[SIC] Successive Interference Cancellation\hspace{1em}串行干扰消除
\item[SINR] Signal-to-Interference-plus-Noise Ratio\hspace{1em}信号与干扰加噪声比
\item[SLAM] Simultaneous Localization and Mapping\hspace{1em}即时定位与地图构建
\item[SNR] Signal-to-Noise Ratio\hspace{1em}信噪比
\item[SR] Service Ratio\hspace{1em}服务率
\item[SV] State-Value\hspace{1em}状态价值
\item[TD] Temporal Difference\hspace{1em}时间差分
\item[uRLLC] ultra-Reliable and Low-Latency Communication\hspace{1em}超高可靠低时延通信
\item[V2C] Vehicle-to-Cloud\hspace{1em}车与云通信
\item[V2I] Vehicle-to-Infrastructure\hspace{1em}车与基础设施通信
\item[V2P] Vehicle-to-Pedestrian\hspace{1em}车与人通信
\item[V2V] Vehicle-to-Vehicle\hspace{1em}车与车通信
\item[V2X] Vehicle-to-Everything\hspace{1em}车联网
\item[VBA] V2I Bandwidth Allocation\hspace{1em}V2I带宽分配
\item[VCCW] View-Calibration-based Collision Warning\hspace{1em}基于视图修正的碰撞预警
\item[VCPS] Vehicular Cyber-Physical System\hspace{1em}车载信息物理融合系统
\item[VEC] Vehicular Edge Computing\hspace{1em}车载边缘计算
\item[VRU] Vulnerable Road User\hspace{1em}弱势道路参与者
\item[WAVE] Wireless Access in Vehicular Environments\hspace{1em}无线接入车载环境
\end{abbreviate}
\endinput